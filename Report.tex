%\documentclass[journal]{IEEEtran}
\documentclass[draftcls,onecolumn,12pt]{IEEEtran}
\IEEEoverridecommandlockouts
%\overrideIEEEmargins

\usepackage[T1]{fontenc}
\usepackage[latin9]{inputenc}
\usepackage[compress]{natbib}
\usepackage{amsthm}
\usepackage{amsmath}
\usepackage{algcompatible}
\usepackage{algorithm}
\usepackage{booktabs}
\usepackage{enumerate}
\usepackage{graphicx}
\usepackage{amssymb}
\usepackage{latexsym}
\usepackage{epstopdf}
\usepackage{color}
\usepackage{bbm}
\usepackage{needspace}
\usepackage{color, colortbl}
\usepackage[english]{babel}
\usepackage{tikz}
\usepackage{caption}
\usepackage{subcaption}
\usetikzlibrary{arrows,automata}
\usetikzlibrary{positioning}
\usepackage{filecontents}

\DeclareCaptionFont{mysize}{\fontsize{8}{9.6}\selectfont}
\captionsetup{font=mysize}

\pagenumbering{gobble}
% *** GRAPHICS RELATED PACKAGES ***
%
\ifCLASSINFOpdf
  % \usepackage[pdftex]{graphicx}
  % declare the path(s) where your graphic files are
  % \graphicsinfectionpath{{../pdf/}{../jpeg/}}
  % and their extensions so you won't have to specify these with
  % every instance of \includegraphics
  % \DeclareGraphicsExtensions{.pdf,.jpeg,.png}
\else
  % or other class option (dvipsone, dvipdf, if not using dvips). graphicx
  % will default to the driver specified in the system graphics.cfg if no
  % driver is specified.
  % \usepackage[dvips]{graphicx}
  % declare the path(s) where your graphic files are
  % \graphicspath{{../eps/}}
  % and their extensions so you won't have to specify these with
  % every instance of \includegraphics
  % \DeclareGraphicsExtensions{.eps}
\fi
% graphicx was written by David Carlisle and Sebastian Rahtz. It is
% required if you want graphics, photos, etc. graphicx.sty is already
% installed on most LaTeX systems. The latest version and documentation can
% be obtained at:
% http://www.ctan.org/tex-archive/macros/latex/required/graphics/
% Another good source of documentation is "Using Imported Graphics in
% LaTeX2e" by Keith Reckdahl which can be found as epslatex.ps or
% epslatex.pdf at: http://www.ctan.org/tex-archive/info/
%
% latex, and pdflatex in dvi mode, support graphics in encapsulated
% postscript (.eps) format. pdflatex in pdf mode supports graphics
% in .pdf, .jpeg, .png and .mps (metapost) formats. Users should ensure
% that all non-photo figures use a vector format (.eps, .pdf, .mps) and
% not a bitmapped formats (.jpeg, .png). IEEE frowns on bitmapped formats
% which can result in "jaggedy"/blurry rendering of lines and letters as
% well as large increases in file sizes.
%
% You can find documentation about the pdfTeX application at:
% http://www.tug.org/applications/pdftex


% *** MATH PACKAGES ***
%
%\usepackage[cmex10]{amsmath}
% A popular package from the American Mathematical Society that provides
% many useful and powerful commands for dealing with mathematics. If using
% it, be sure to load this package with the cmex10 option to ensure that
% only type 1 fonts will utilized at all point sizes. Without this option,
% it is possible that some math symbols, particularly those within
% footnotes, will be rendered in bitmap form which will result in a
% document that can not be IEEE Xplore compliant!
%
% Also, note that the amsmath package sets \interdisplaylinepenalty to 10000
% thus preventing page breaks from occurring within multiline equations. Use:
%\interdisplaylinepenalty=2500
% after loading amsmath to restore such page breaks as IEEEtran.cls normally
% does. amsmath.sty is already installed on most LaTeX systems. The latest
% version and documentation can be obtained at:
% http://www.ctan.org/tex-archive/macros/latex/required/amslatex/math/

\addtolength{\textwidth}{-6mm}
\addtolength{\hoffset}{3mm}
\addtolength{\textheight}{-0mm}
\addtolength{\voffset}{4mm}
\theoremstyle{plain}
\newtheorem{thm}{\protect\theoremname}
  \theoremstyle{plain}
  \newtheorem{lem}[thm]{\protect\lemmaname}
\newtheorem{remark}{Remark}
\newtheorem{coro}{Corollary}
\newtheorem{conjecture}{Conjecture}
\newtheorem{define}{Definition}
\newtheorem{problem}{Problem}


\makeatother

\usepackage{babel}
\providecommand{\lemmaname}{Lemma}
\providecommand{\theoremname}{Theorem}
\newcommand{\norm}[1]{\left\lVert#1\right\rVert}
\renewcommand{\algorithmicrequire}{\textbf{Input:}}
\renewcommand{\algorithmicensure}{\textbf{Output:}}
\newcommand{\Sysbi}{$\mathcal B(\bar A,\bar B)$}
\newcommand{\Digraph}{$G(\bar A,\bar B)$}

\renewcommand{\thesubsubsection}{\alph{subsubsection})}
%\renewcommand{\baselinestretch}{1.5}


\begin{document}
\title{}
\author{%
\thanks{The authors are with the Department of Electrical and Systems Engineering
at the University of Pennsylvania, Philadelphia PA 19104.  %
}}
\maketitle

\begin{abstract}
\end{abstract}

\section{Introduction}

\section{Notation and Preliminaries}\label{sec:prelim}



%%%%%%%%%%%%%%%%%%%%%%%%%%%%%%%%% Figure %%%%%%%%%%%%%%%%%%%%%%%%%%%%%%%%%%%%%%%%%%%%%%%
%\begin{figure}[htb!!]
%    \centering
%    \begin{subfigure}[t]{0.45\textwidth}
%        \includegraphics[width=\textwidth,height=\textwidth]{./images/.pdf}\\
%        \caption{}
%    \end{subfigure}
%    \hspace{-0.5cm}
%    \begin{subfigure}[t]{0.45\textwidth}
%        \includegraphics[width=\textwidth,height=\textwidth]{./images/.pdf}\\
%        \caption{}
%    \end{subfigure}
%    \caption{. }\label{Fig:}
%\end{figure}




%%%%%%%%%%%%%%%%%%%%%%%%%%%%%%%%% Table %%%%%%%%%%%%%%%%%%%%%%%%%%%%%%%%%%%%%%%%%%%%%%%%
%\begin{table}[!h]
%\centering
%\begin{tabular}{@{} *4l @{}}    \toprule
%  \emph{Name of Network} & \emph{Actual \# Pentagons} & \emph{Estimated \# of Pentagons}  & \emph{Percentage Estimation Error} \\ \midrule
%  Health Network ($n = 2539$) & 13874 & 47929 & $>$ 100\% \\  \bottomrule \hline
%\end{tabular}\caption{This table shows the comparison between actual number and estimated number of pentagons in different real networks.}\label{Tab:PercentageError}
%\end{table}


%%%%%%%%%%%%%%%%%%%%%%%%%%%%%%%%% Algorithm %%%%%%%%%%%%%%%%%%%%%%%%%%%%%%%%%%%%%%%%%%%%
%\begin{algorithm}
%\caption{}\label{alg:}
%\begin{algorithmic}[1]
%\REQUIRE
%\ENSURE
%\STATEx
%\STATE
%\IF{}
%    \STATE
%\ELSE
%    \STATE
%\ENDIF
%\algstore{bkbreak}
%\end{algorithmic}
%\end{algorithm}

%\begin{algorithm}
%\begin{algorithmic}[1]
%\algrestore{bkbreak}
%\STATEx
%\STATE
%\IF{}
%    \FOR{}
%        \STATE $k \gets k+1.$
%    \ENDFOR
%\ELSE
%\ENDIF
%\end{algorithmic}
%\end{algorithm}



%%%%%%%%%%%%%%%%%%%%%%%%%%%%%%%%%% APPENDIX %%%%%%%%%%%%%%%%%%%%%%%%%%%%%%%%%%%%%%%%%%
%\begin{appendices}
%\section{}\label{appen:proof}
%\input{appendix_proof}
%\end{appendices}


%%%%%%%%%%%%%%%%%%%%%%%%%%%%%%%%%% Reference %%%%%%%%%%%%%%%%%%%%%%%%%%%%%%%%%%%%%%%%
\small
\bibliographystyle{IEEEtran}
{\small \bibliography{reference}}

\end{document}

